\documentclass[11pt]{amsart}
\usepackage{calc,amssymb,amsthm,amsmath,fullpage}
% \usepackage{mathtools}
\RequirePackage[dvipsnames,usenames, table]{xcolor}
\usepackage{hyperref}
\usepackage[capitalize]{cleveref}
\hypersetup{
  bookmarks,
  bookmarksdepth=3,
  bookmarksopen,
  bookmarksnumbered,
  pdfstartview=FitH,
  colorlinks,backref,hyperindex,
  linkcolor=Sepia,
  anchorcolor=BurntOrange,
  citecolor=MidnightBlue,
  citecolor=OliveGreen,
  filecolor=BlueViolet,
  menucolor=Yellow,
  urlcolor=OliveGreen
}
\usepackage{xypic}
\usepackage{alltt}
\usepackage{multicol}  
% \usepackage{etex}
\usepackage{xspace}
\usepackage{rotating}
\interfootnotelinepenalty=100000
\theoremstyle{definition}
\newtheorem{example}{Example}[section]
\newtheorem{remark}{Remark}[subsection]

\newcommand{\bA}{\mathbb{A}}
\newcommand{\bF}{\mathbb{F}}
%\usepackage{enumitem}
%\SetLabelAlign{parleft}{\parbox[t]\textwidth{#1\smallskip}}

% \usepackage{mabliautoref}
% \usepackage{colonequals}
% \frenchspacing
% %%   Setup to use Ralph Smith Formal Script font:
%%   Provided by Sandor Kovacs to Karl Schwede, modified further
\DeclareFontFamily{OMS}{rsfs}{\skewchar\font'60}
\DeclareFontShape{OMS}{rsfs}{m}{n}{<-5>rsfs5 <5-7>rsfs7 <7->rsfs10 }{}
\DeclareSymbolFont{rsfs}{OMS}{rsfs}{m}{n}
\DeclareSymbolFontAlphabet{\scr}{rsfs}

\newenvironment{pf}[1][{\it Proof:}]{\begin{trivlist}
\item[\hskip \labelsep {\bfseries #1}]}{\end{trivlist}}

%%***Collaborative editing commands***
\newcommand{\note}[1]{\marginpar{\sffamily\tiny #1}}

\def\todo#1{\textcolor{Mahogany}%
{\sffamily\footnotesize\newline{\color{Mahogany}\fbox{\parbox{\textwidth-15pt}{#1}}}\newline}}


%%***Common Names***
\newcommand{\Cech}{{$\check{\text{C}}$ech} }
\newcommand{\mustata}{Musta{\c{t}}{\u{a}}}
\newcommand{\etale}{{\'e}tale }


%%%************Functors, derived categories and related****************
\newcommand{\myR}{{\bf R}}
\newcommand{\myH}{{\bf h}}
\newcommand{\myL}{{\bf L}}
\newcommand{\qis}{\simeq_{\text{qis}}}
\newcommand{\tensor}{\otimes}
\newcommand{\hypertensor}{{\uuline \tensor}}
\newcommand{\mydot}{{{\,\begin{picture}(1,1)(-1,-2)\circle*{2}\end{picture}\ }}}
\newcommand{\blank}{\underline{\hskip 10pt}}
\DeclareMathOperator{\trunc}{{trunc}}

%%*************Diagram of scheme notation
\DeclareMathOperator{\kSch}{{Sch}}
\DeclareMathOperator{\kCat}{{Cat}}
\DeclareMathOperator{\kHrc}{{Hrc}}
\newcommand{\uSch}{\underline{\kSch}}
\newcommand{\uCat}{\underline{\kCat}}
\newcommand{\uHrc}{\underline{\kHrc}}
\newcommand{\Ho}{\text{Ho}}

%%%*************Du Bois short hand******************
\newcommand{\DuBois}[1]{{\uuline \Omega {}^0_{#1}}}
\newcommand{\FullDuBois}[1]{{\uuline \Omega {}^{\mydot}_{#1}}}

\newcommand{\modGens}[2]{{{\bf \mu}_{#1}{\left(#2\right)}}}


%%%*************F-singularity short hand************
\newcommand{\tauCohomology}{T}
\newcommand{\FCohomology}{S}
\newcommand{\Ram}{\mathrm{Ram}}
\DeclareMathOperator{\Tr}{Tr}
\newcommand{\roundup}[1]{\lceil #1 \rceil}
\newcommand{\rounddown}[1]{\lfloor #1 \rfloor}
\newcommand{\HK}{\mathrm{HK}}

%%***************Latin short hand*******************
\newcommand{\cf}{{\itshape cf.} }
\newcommand{\loccit}{{\itshape loc. cit.} }
\newcommand{\ie}{{\itshape i.e.} }

%%************Script**********************
\newcommand{\sA}{\scr{A}}
\newcommand{\sB}{\scr{B}}
\newcommand{\sC}{\scr{C}}
\newcommand{\sD}{\scr{D}}
\newcommand{\sE}{\scr{E}}
\newcommand{\sF}{\scr{F}}
\newcommand{\sG}{\scr{G}}
\newcommand{\sH}{\scr{H}}
\newcommand{\sI}{\scr{I}}
\newcommand{\sJ}{\scr{J}}
\newcommand{\sK}{\scr{K}}
\newcommand{\sL}{\scr{L}}
\newcommand{\sM}{\scr{M}}
\newcommand{\sN}{\scr{N}}
\newcommand{\sO}{\scr{O}}
\newcommand{\sP}{\scr{P}}
\newcommand{\sQ}{\scr{Q}}
\newcommand{\sR}{\scr{R}}
\newcommand{\sS}{\scr{S}}
\newcommand{\sT}{\scr{T}}
\newcommand{\sU}{\scr{U}}
\newcommand{\sV}{\scr{V}}
\newcommand{\sW}{\scr{W}}
\newcommand{\sX}{\scr{X}}
\newcommand{\sY}{\scr{Y}}
\newcommand{\sZ}{\scr{Z}}

\newcommand{\enm}[1]{\ensuremath{#1}}
\newcommand{\mJ}{\mathcal{J}}
\newcommand{\uTwo}{\underline{2}}
\newcommand{\uOne}{\underline{1}}
\newcommand{\uZero}{\underline{0}}

\newcommand{\ba}{\mathfrak{a}}
\newcommand{\bb}{\mathfrak{b}}
\newcommand{\bc}{\mathfrak{c}}
\newcommand{\bff}{\mathfrak{f}}
\newcommand{\bm}{\mathfrak{m}}
\newcommand{\bn}{\mathfrak{n}}
\newcommand{\bp}{\mathfrak{p}}
\newcommand{\bq}{\mathfrak{q}}
\newcommand{\bt}{\mathfrak{t}}
\newcommand{\fra}{\mathfrak{a}}
\newcommand{\frb}{\mathfrak{b}}
\newcommand{\frc}{\mathfrak{c}}
\newcommand{\frf}{\mathfrak{f}}
\newcommand{\fram}{\mathfrak{m}}
%\renewcommand{\frm}{\mathfrak{m}}
\newcommand{\frn}{\mathfrak{n}}
\newcommand{\frp}{\mathfrak{p}}
\newcommand{\frq}{\mathfrak{q}}
\newcommand{\frt}{\mathfrak{t}}

\newcommand{\bA}{\mathbb{A}}
\newcommand{\bB}{\mathbb{B}}
\newcommand{\bC}{\mathbb{C}}
\newcommand{\bD}{\mathbb{D}}
\newcommand{\bE}{\mathbb{E}}
\newcommand{\bF}{\mathbb{F}}
\newcommand{\bG}{\mathbb{G}}
\newcommand{\bH}{\mathbb{H}}
\newcommand{\bI}{\mathbb{I}}
\newcommand{\bJ}{\mathbb{J}}
\newcommand{\bK}{\mathbb{K}}
\newcommand{\bL}{\mathbb{L}}
\newcommand{\bM}{\mathbb{M}}
\newcommand{\bN}{\mathbb{N}}
\newcommand{\bO}{\mathbb{O}}
\newcommand{\bP}{\mathbb{P}}
\newcommand{\bQ}{\mathbb{Q}}
\newcommand{\bR}{\mathbb{R}}
\newcommand{\bS}{\mathbb{S}}
\newcommand{\bT}{\mathbb{T}}
\newcommand{\bU}{\mathbb{U}}
\newcommand{\bV}{\mathbb{V}}
\newcommand{\bW}{\mathbb{W}}
\newcommand{\bX}{\mathbb{X}}
\newcommand{\bY}{\mathbb{Y}}
\newcommand{\bZ}{\mathbb{Z}}


\newcommand{\al}{\alpha}
\newcommand{\be}{\beta}
\newcommand{\ga}{\gamma}
\newcommand{\de}{\delta}
\newcommand{\pa}{\partial}   %pretend its Greek
\newcommand{\epz}{\varepsilon}
\newcommand{\ph}{\phi}
\newcommand{\phz}{\varphi}
\newcommand{\et}{\eta}
\newcommand{\io}{\iota}
\newcommand{\ka}{\kappa}
\newcommand{\la}{\lambda}
\newcommand{\tha}{\theta}
\newcommand{\thz}{\vartheta}
\newcommand{\rh}{\rho}
\newcommand{\si}{\sigma}
\newcommand{\ta}{\tau}
\newcommand{\ch}{\chi}
\newcommand{\ps}{\psi}
\newcommand{\ze}{\zeta}
\newcommand{\om}{\omega}
\newcommand{\GA}{\Gamma}
\newcommand{\LA}{\Lambda}
\newcommand{\DE}{\Delta}
\newcommand{\SI}{\Sigma}
\newcommand{\THA}{\Theta}
\newcommand{\OM}{\Omega}
\newcommand{\XI}{\Xi}
\newcommand{\UP}{\Upsilon}
\newcommand{\PI}{\Pi}
\newcommand{\PS}{\Psi}
\newcommand{\PH}{\Phi}

\newcommand{\com}{\circ}     % composition of functions
\newcommand{\iso}{\simeq}    % preferred isomorphism symbol
\newcommand{\ten}{\otimes}   % tensor product
\newcommand{\add}{\oplus}    % direct sum

\newcommand{\ul}{\underline}
\newcommand{\nsubset}{\not\subset}
\newcommand{\tld}{\widetilde }
\renewcommand{\:}{\colon}


\newcommand{\rtarr}{\longrightarrow}
\newcommand{\ltarr}{\longleftarrow}
\newcommand{\from}{\longleftarrow}
\newcommand{\monoto}{\lhook\joinrel\relbar\joinrel\rightarrow}
\newcommand{\epito}{\relbar\joinrel\twoheadrightarrow}

%%%%%%%%% math short hand
%%%% gothic
\newcommand{\Schs}{\mathfrak S\mathfrak c\mathfrak h_{S}}
\newcommand{\LocFrees}{\mathfrak L\mathfrak o\mathfrak c\mathfrak F\mathfrak
 r\mathfrak e\mathfrak e_{S}}
\newcommand{\A}{\mathfrak A}
\newcommand{\Ab}{\mathfrak A\mathfrak b}
\newcommand{\B}{\mathfrak B}
\newcommand{\M}{\mathfrak M\mathfrak o\mathfrak d}
\newcommand{\Mg}{\mathfrak M_g}
\newcommand{\Mgbar}{\overline{\mathfrak M}_g}
\newcommand{\Mh}{\mathfrak M_h}
\newcommand{\Mhbar}{\overline{\mathfrak M}_h}
\newcommand{\maxm}{\mathfrak m}

%%%% curly
\newcommand{\m}{\scr M}
\newcommand{\n}{\scr N}
\newcommand{\cO}{\mathcal O}
\renewcommand{\O}{\mathcal O}
\newcommand{\f}{\scr F}
\renewcommand{\O}{\scr O}
\newcommand{\I}{\scr I}
\newcommand{\J}{\scr{J}}

%%%% Blackboard bold
\newcommand{\C}{\mathbb {C}}
\newcommand{\N}{\mathbb {N}}
\newcommand{\R}{\mathbb {R}}
\newcommand{\PP}{\mathbb {P}}
\newcommand{\Z}{\mathbb {Z}}
\newcommand{\Q}{\mathbb {Q}}
\renewcommand{\r}{\mathbb R^{+}}
\newcommand{\NZ}{\mbox{$\mathbb{N}$}}
\renewcommand{\O}{\mbox{$\mathcal{O}$}}
\renewcommand{\P}{\mathbb{P}}
\newcommand{\ZZ}{\mbox{$\mathbb{Z}$}}
%%%%
\newcommand{\infinity}{\infty}
\newcommand{\ney}{\overline{NE}(Y)}
\newcommand{\nex}{\overline{NE}(X)}
\newcommand{\nes}{\overline{NE}(S)}
%%%%
\newcommand{\sub}{\subseteq}
\newcommand{\ratmap}{\dasharrow}
\newcommand{\eq}{\equiv}
\newcommand{\myquad}{\ }
%%%
%%%%%%% operators
\DeclareMathOperator{\Char}{{char}}
\DeclareMathOperator{\Cart}{{Cartier}}
\DeclareMathOperator{\fpt}{{fpt}}
\DeclareMathOperator{\divisor}{{div}}
\DeclareMathOperator{\Div}{{div}}
\DeclareMathOperator{\ord}{{ord}}
\DeclareMathOperator{\Frac}{{Frac}}
\DeclareMathOperator{\Ann}{{Ann}}
\DeclareMathOperator{\rd}{{rd}}
\DeclareMathOperator{\an}{{an}}
\DeclareMathOperator{\height}{{ht}}
\DeclareMathOperator{\exc}{{exc}}
\DeclareMathOperator{\coherent}{{coh}}
\DeclareMathOperator{\quasicoherent}{{qcoh}}
\DeclareMathOperator{\sn}{{sn}}
\DeclareMathOperator{\wn}{{wn}}
\DeclareMathOperator{\id}{{id}}
\DeclareMathOperator{\codim}{codim}
\DeclareMathOperator{\coker}{{coker}}
%%\DeclareMathOperator{\ker}{{ker}}
\DeclareMathOperator{\im}{{im}}
\DeclareMathOperator{\sgn}{{sgn}}
%%\DeclareMathOperator{\hom}{{Hom}}
\DeclareMathOperator{\opp}{{op}}
\DeclareMathOperator{\ext}{{Ext}}
\DeclareMathOperator{\Tor}{{Tor}}
\DeclareMathOperator{\pic}{{Pic}}
\DeclareMathOperator{\pico}{{Pic}^{\circ}}
\DeclareMathOperator{\aut}{{Aut}}
\DeclareMathOperator{\bir}{{Bir}}
\DeclareMathOperator{\lin}{{Lin}}
\DeclareMathOperator{\sym}{{Sym}}
\DeclareMathOperator{\rank}{{rank}}
\DeclareMathOperator{\rk}{{rk}}
\DeclareMathOperator{\pgl}{{PGL}}
\DeclareMathOperator{\gl}{{GL}}
\DeclareMathOperator{\Gr}{{Gr}}
\DeclareMathOperator{\ob}{{Ob}}
\DeclareMathOperator{\mor}{{Mor}}
\DeclareMathOperator{\supp}{{supp}}
\DeclareMathOperator{\Supp}{{Supp}}
\DeclareMathOperator{\Sing}{{Sing}}
\DeclareMathOperator{\var}{{Var}}
\DeclareMathOperator{\Spec}{{Spec}}
\DeclareMathOperator{\Proj}{{Proj}}
\DeclareMathOperator{\Tot}{{Tot}}
\DeclareMathOperator{\Aut}{Aut}
\DeclareMathOperator{\Lef}{Lef}
\DeclareMathOperator{\wt}{wt}
\DeclareMathOperator{\twoRC}{{RC_2^n}}
\DeclareMathOperator{\ptRC}{{RC_{\bullet}}}
\DeclareMathOperator{\twoptRC}{{RC^2_{\bullet}}}
\DeclareMathOperator{\Univ}{Univ}
\DeclareMathOperator{\Univrc}{{Univ^{rc}}}
\DeclareMathOperator{\twoUnivrc}{{Univ^{rc, 2}}}
\DeclareMathOperator{\ptUnivrc}{{Univ^{rc}_{\bullet}}}
\DeclareMathOperator{\twoptUnivrc}{{Univ_{\bullet}^{rc, 2}}}
\DeclareMathOperator{\charact}{char}
\DeclareMathOperator{\Chow}{Chow}
\DeclareMathOperator{\Dubbies}{Dubbies^n}
\DeclareMathOperator{\Ext}{Ext}
\DeclareMathOperator{\Hilb}{Hilb}
\DeclareMathOperator{\Hom}{Hom}
\DeclareMathOperator{\sHom}{{\sH}om}
\DeclareMathOperator{\Hombir}{Hom_{bir}^n}
\DeclareMathOperator{\Image}{Image}
\DeclareMathOperator{\genus}{genus}
\DeclareMathOperator{\Imaginary}{Im}
\DeclareMathOperator{\Img}{Im}
\DeclareMathOperator{\Ker}{Ker}
\DeclareMathOperator{\locus}{locus}
\DeclareMathOperator{\Num}{Num}
\DeclareMathOperator{\Pic}{Pic}
\DeclareMathOperator{\RatCurves}{RatCurves^n}
\DeclareMathOperator{\RC}{RatCurves^n}
\DeclareMathOperator{\red}{red}
\DeclareMathOperator{\Reg}{Reg}
\DeclareMathOperator{\psl}{PGL}
\DeclareMathOperator{\mult}{mult}
\DeclareMathOperator{\mld}{mld}
\renewcommand{\mod}[1]{\,(\textnormal{mod}\,#1)}
%%%%%%%%%%%%%%%%%%%%%%%%%%%%%%%%%%%%%
\def\spec#1.#2.{{\bold S\bold p\bold e\bold c}_{#1}#2}
\def\proj#1.#2.{{\bold P\bold r\bold o\bold j}_{#1}\sum #2}
\def\ring#1.{\scr O_{#1}}
\def\map#1.#2.{#1 \to #2}
\def\longmap#1.#2.{#1 \longrightarrow #2}
\def\factor#1.#2.{\left. \raise 2pt\hbox{$#1$} \right/
\hskip -2pt\raise -2pt\hbox{$#2$}}
\def\pe#1.{\mathbb P(#1)}
\def\pr#1.{\mathbb P^{#1}}
\newcommand{\sheafspec}{\mbox{\bf Spec}}
\newcommand{\len}[2]{{{\bf \ell}_{#1}{\left(#2\right)}}}

%%%%%%%%%%%%%%%%%%%%%%%%%%%%%%%%%%%%%%%%%%%%%%%%%
%%%%%% cohomology and short exact sequences %%%%%
%%%%%%%%%%%%%%%%%%%%%%%%%%%%%%%%%%%%%%%%%%%%%%%%%
\def\coh#1.#2.#3.{H^{#1}(#2,#3)}
\def\dimcoh#1.#2.#3.{h^{#1}(#2,#3)}
\def\hypcoh#1.#2.#3.{\mathbb H_{\vphantom{l}}^{#1}(#2,#3)}
\def\loccoh#1.#2.#3.#4.{H^{#1}_{#2}(#3,#4)}
\def\dimloccoh#1.#2.#3.#4.{h^{#1}_{#2}(#3,#4)}
\def\lochypcoh#1.#2.#3.#4.{\mathbb H^{#1}_{#2}(#3,#4)}
%%%%%%%%%%
\def\ses#1.#2.#3.{0  \longrightarrow  #1   \longrightarrow
 #2 \longrightarrow #3 \longrightarrow 0}
\def\sesshort#1.#2.#3.{0
 \rightarrow #1 \rightarrow #2 \rightarrow #3 \rightarrow 0}
 \def\sesa#1{0
 \rightarrow #1 \rightarrow #1 \rightarrow #1 \rightarrow 0}

\renewcommand{\to}[1][]{\xrightarrow{\ #1\ }}
\newcommand{\onto}[1][]{\protect{\xrightarrow{\ #1\ }\hspace{-0.8em}\rightarrow}}
\newcommand{\into}[1][]{\lhook \joinrel \xrightarrow{\ #1\ }}
%%%%%%%%%%
%%%%%%%%%% iff
\def\iff#1#2#3{
    \hfil\hbox{\hsize =#1 \vtop{\noin #2} \hskip.5cm
    \lower.5\baselineskip\hbox{$\Leftrightarrow$}\hskip.5cm
    \vtop{\noin #3}}\hfil\medskip}
%%%%%%%%%%%%%%%%%%%%%%%%%%%%%%
\def\myoplus#1.#2.{\underset #1 \to {\overset #2 \to \oplus}}
\def\assign{\longmapsto}
%%%%%%%%%%%%%%%%%%%%%%%%%%%%%%
%%%%%%%%%%%%%%%%%%%%%%%%%%%%%%%%%%
%%% Arrows %%%%%%%%%%%%%%
%%%%%%%%%%%%%%%%%%%%%%%%%%%%%%%%%%%

% \usepackage{stmaryrd}
% 
\usepackage{verbatim}
\usepackage{enumerate}
 
\begin{document}
\title{{RandomRationalPoints} package for \emph{Macaulay2}}
\author{Sankhaneel Bisui}
\email{sbisui@tulane.edu}
\address{Department of Mathematics, Tulane University, New Orleans, LA 70118}
\author{Zhan Jiang}
\email{zoeng@umich.edu}
\address{Department of Mathematics, University of Michigan, Ann Arbor, MI 48109}
\author{Sarasij Maitra}
\email{sm3vg@virginia.edu}
\address{Department of Mathematics, University of Virginia, Charlottesville, VA 22904}
\author{Th\'ai Th\`anh Nguy$\tilde{\text{\^e}}$n}
\email{tnguyen11@tulane.edu}
\address{Department of Mathematics, Tulane University, New Orleans, LA 70118}
\author{Karl Schwede}
\email{schwede@math.utah.edu}
\address{Department of Mathematics, University of Utah, 155 S 1400 E Room 233, Salt Lake City, UT, 84112}
%\date{\today}

\thanks{Schwede was supported by NSF Grants \#1801849, FRG \#1952522 and a Fellowship from the Simons Foundation.}

\begin{abstract}
  We present {\tt RandomRationalPoints}, a package in \emph{Macaulay2} designed mainly to identify rational and geometric points in a variety over a finite field. We provide different strategies such as linear intersection and projection to obtain such points. We also present methods to obtain non-vanishing minors of a given size in a given matrix, by evaluating the matrix at a point.  %This, in conjunction with the {\tt FastLinAlg} package, should aid in determining properties of the singular locus.
\end{abstract}


\keywords{RandomRationalPoints, Macaulay2}
% \thanks{Martinova was supported by a University of Utah Mathematics REU fellowship and by the University of Utah ACCESS program.  
%% Yao was supported by a University of Utah Mathematics REU fellowship.}

\maketitle

\section{Introduction}
    Let $I$ be an ideal in a polynomial ring $k[x_1,\dots, x_n]$ over a finite field $k$. Let $X:=V(I)$ denote the corresponding set of rational points in affine $n$-space. Finding one such rational point or geometric point (geometric meaning a point in some finite field extension), in an algorithmically efficient manner is our primary motivation. 

    There is an existing package called {\tt RationalPoints} \cite{RationalPointsSource}, which we took inspiration from, which aims to find \emph{all} the rational points of a variety; our aim here is to find one or more points quickly, even if it is not rational.  We note also that the package {\tt Cremona} \cite{CremonaSource} can find rational points on projective varieties, as can built-in the function {\tt randomKRationalPoint} \cite{M2}, our methods frequently appear to be faster and apply in the affine setting as well.

    We develop functions that apply different strategies to generate random rational and geometric points on the given variety. We also provide functions that will expedite the process of determining properties of the singular locus of $X$.% when combined with the functions provided in the package {\tt FastLinAlg}.  

    We provide the following core functions:
    \begin{itemize}    
    \item {\tt randomPoints}:  This tries to find a point in the vanishing set of an ideal. (\Cref{randomPoints})
    \item {\tt projectionToHypersurface} and {\tt genericProjection}: These functions provide customizable projection. (\Cref{projectionfunctions}) 	
    \item {\tt findANonZeroMinor} and {\tt extendIdealByNonZeroMinor}:  The first function finds a submatrix of a given matrix that is nonsingular at a point of a given ideal.  The second, adds said submatrix to an ideal, which is useful for computing partial Jacobian ideals.  (\Cref{findANonZeroMinor})          
    %\item :  This extends an ideal by a minor produced by \\{\tt findANonZeroMinor}. (\Cref{extendIdealByNonZeroMinor})	
    \end{itemize}

    All polynomial rings considered here will be over finite fields. In the subsequent sections, we explain  the core and helper functions and describe the strategies that we have implemented.

   \vspace{1em}
    \noindent \textbf{Acknowledgements:} The authors would like to thank David Eisenbud and Mike Stillman for useful conversations and comments on the development of this package.  The authors began work on this package at the virtual Cleveland 2020 Macaulay2 workshop. The authors are also grateful to the reviewers for suggesting and providing preliminary codes to speed up computations, thereby improving the efficacy of the package substantially. 

\section{Our primary purpose: {\tt randomPoints}}\label{randomPoints}
    We start with the core function in this package:  {\tt randomPoints} is a function to find rational or geometric points in a variety. The typical usages are as follows: 

    \vspace{0.5em}
    -- {\tt randomPoints(I)}, 

    -- {\tt randomPoints(n, I)} 

    \vspace{0.5em}
    \noindent where $n$ is a positive integer denoting the number of desired points, and 
    $I$ is an ideal inside a polynomial ring.  If {$ n$} is omitted, then it is assumed to be 1.


    \subsection{Options}\label{strategydetails}

    The user may also choose to provide some additional information depending on the context which may help in faster computations, or whether a point is found at all.

    \begin{description}%[]
        \setlength{\itemsep}{5pt}
    \item[\tt Strategy => $\bullet$]   Here the $\bullet$ can be {\tt Default, BruteForce, LinearIntersection, \\ GenericProjection}, {\tt HybridProjectionIntersection} or {\tt MultiplicationTable}.

    \begin{itemize}
    \item {\tt Default} performs a sequence of the different strategies below, aimed at finding a point quickly.  It begins begins with brute force and moves to linear intersections with particularly simple linear forms, and gradually ramps up the randomness.  If {\tt ProjectionAttempts} is set to a value bigger than zero (it is set to zero by default), then the strategy alternates between generic projects and hybrid projection / intersections.  For additional discussion, see \cref{rem.ProjectingToAHypersurfaceFirst}.  
    \item {\tt BruteForce} simply tries random points and sees if they are on the variety.
        
    \item {\tt GenericProjection} projects to a hypersurface, via 
        {\tt projectionToHypersurface} and then uses the {\tt BruteForce} strategy.  Finally, it pulls back back the points that were found to the original variety (via a linear intersection).  See \cref{rem.ProjectingToAHypersurfaceFirst}.  
        
    \item {\tt LinearIntersection} intersects with an appropriately random linear space.
        
    \item {\tt HybridProjectionIntersection} does a generic projection, followed by a linear intersectio with that hypersurface.  Finally, it pulls back back the points that were found to the original variety (via a linear intersection).   See \cref{rem.ProjectingToAHypersurfaceFirst}.  
    \item {\tt MultiplicationTable} works for homogeneous ideals only. It cuts down with a linear space to a zero-dimensional projective scheme and computes how the last two variables act on the quotient ring truncated at the regularity.  %In case of an absolutely irreducible $I$ over the field, we will find with  high probability a point, since the determinant will have a linear factor in at least 50\% of the cases.
    \end{itemize}

    Notice that the speed, or success, varies depending on the strategy (see also \Cref{helper}).

    \begin{example}\label{BruteForce}
        Consider the following example.
        ~~        
        {\small\color{blue}
    \begin{quote}
\begin{verbatim}
i2 : R = ZZ/101[x, y, z];

i3 : J = ideal(x^3 + y^2 + 1, z^3 - x^2 - y^2 + 2);

o3 : Ideal of R

i4 : randomPoints(J)
    -- used 0.0181549 seconds
    
o4 = {{32, 37, -30}}

o4 : List

i5 : time randomPoints(J,Strategy=>BruteForce)
    -- used 0.0146932 seconds
    
o5 = {}

o5 : List

i6 : randomPoints(J,Strategy=>GenericProjection)
    -- used 0.0146932 seconds
    
o6 = {{2, -30, -8}}
        \end{verbatim}
    \end{quote}
        }
    \end{example}%
    %
    \vspace{-1em}
    \item[\tt ProjectionAttempts => ZZ] 

    When calling the strategy {\tt GenericProjection} or \\
    {\tt HybridProjectionIntersection} from {\tt randomPoints}, this option denotes the number of trials before giving up.  This option is also passed to {\tt randomPoints} by other functions.

    \item[\tt MaxCoordinatesToReplace => ZZ] This is used for fine tuning the {\tt LinearIntersection} or {\tt GenericProjection} strategies, see \Cref{helper}.
    \item[\tt Codimension => ZZ] One can provide the codimension so it is not recomputed.  This is used in the {\tt GenericProjection} and {\tt LinearIntersection} strategies, see \Cref{projectionfunctions}. 

    \item[\tt ExtendField => Boolean] 

    Intersection with a general linear space will naturally find scheme theoretic points that are not rational over the base field.  
    Setting {\tt ExtendField => true} will tell the function that such points are valid.  Setting {\tt ExtendField => false} will tell the function ignore such points.  This sometimes can slow computation, and other times can substantially speed it up when the variety has few rational points.  In some cases, points over extended fields may also have better randomness properties for applications.


    \item[\tt IntersectionAttempts => ZZ]

    This option is used by {\tt randomPoints} in some strategies to determine the maximum number of attempts to intersect with a linear space when looking for random rational points.  %Other functions pass this option through to {\tt randomPoints}.

    \item[\tt PointCheckAttempts => ZZ]

    When calling {\tt randomPoints} with a {\tt BruteForce} strategy, this denotes the number of trials for brute force point checking.
    
    \begin{example}
        We re-compute \cref{BruteForce} this time specifying more attempts.
    {{\small\color{blue}
    ~~
    \begin{quote}
\begin{verbatim}    
i7 : randomPoints(J, Strategy => BruteForce, PointCheckAttempts => 20000)
    -- used 0.84679 seconds
    
o7 = {{-39, -43, 28}}
    \end{verbatim}%        
\end{quote}\vspace{-1em}%
}}%
    \end{example}%    
%
    %\vspace{1em}
    \item[\tt MaxCoordinatesToTrivialize]

    When calling {\tt randomPoints} and performing an intersection with a linear space, this is the number of defining equations of the linear space of the form $x_i - a_i$.  Having a large number of these will provide faster intersections.
    
%    \vspace{1em}
    \item[\tt NumThreadsToUse => ZZ]

    When calling {\tt randomPoints} with a {\tt BruteForce} strategy, this denotes the number of threads to use in brute force point checking.
    \end{description}
    % Outputs
    % :List
    % a list of points in the variety with possible repetitions.
    % Description
    % Text  
    % Gives at most $n$ many point in a variety $V(I)$. 
    % Example
    % R = ZZ/5[t_1..t_3];
    % I = ideal(t_1,t_2+t_3);
    % randomPoints(3, I)
    % randomPoints(4, I, Strategy => Default)
    % randomPoints(4, I, Strategy => GenericProjection)
    % randomPoints(4, I, Strategy => LinearIntersection)

    \subsection{Comments on performance}

    When working over very small fields, frequently {\tt BruteForce} is most efficient.  This is not surprising as there may not be many points to check.  However, if the field size is larger, {\tt BruteForce} will perform poorly.  Even for some simple examples, it could not provide any rational points if the number of trials is not enough.  Other strategies work differently on different examples, and the same strategy can sometimes work very quickly even if it typically works very slowly.  
    
    Providing {\tt codimension} speeds up computations significantly as computing codimension using the default command {\tt codim} creates a bottleneck in some cases. If codimension needs to be computed, we  circumvent this issue using a probabilistic code to compute it (see \Cref{helper}). A quick comment, while in many examples generic projection works well, there are some examples where it is many orders of magnitude slower.  By default, generic projection is turned off in the default strategy, that is we set {\tt ProjectionAttempts=>0}.  See \autoref{rem.ProjectingToAHypersurfaceFirst} for additional related discussion.


    \begin{example}
        We begin with an example over a small field.
    {{\small\color{blue}
    ~~
    \begin{quote}
\begin{verbatim}
i2 : R = ZZ/7[x_1..x_10];

i3 : I = ideal(random(2, R), random(3, R));

o3 : Ideal of R

i4 : time randomPoints(I, Strategy => BruteForce, PointCheckAttempts => 20000)
    -- used 0.0102075 seconds
    
o4 = {{-2, -1, 3, -2, -2, 2, 0, -2, -2, -2}}

o4 : List

i5 : time randomPoints(I, Strategy => Default)
    -- used 0.085741 seconds
    
o5 = {{1, -2, 2, 2, -2, 0, 0, -1, -1, 3}}
    \end{verbatim}
\end{quote}\vspace{-1em}
    }}
    \end{example}    

\begin{example}
    Now we work over a larger field.
    {{\small\color{blue}
    \begin{quote}
\begin{verbatim}
i6 : S = ZZ/211[x_1..x_10];

i7 : J = ideal(random(2, S), random(3, S));

o7 : Ideal of S

i8 : time randomPoints(J, Strategy => BruteForce, PointCheckAttempts => 2000000)
    -- used 25.3479 seconds
    
o8 = {{-36, 26, -63, -7, -48, 78, -44, -22, 105, 39}}

i9 : time randomPoints(J, Strategy => Default)
-- used 0.102918 seconds

o9 = {{0, -42, -53, -33, -65, -27, -105, -37, 99, 64}}

o9 : List

i10 : time randomPoints(J, Strategy => LinearIntersection)
-- used 0.0169781 seconds

o10 = {{-51, -11, -34, -33, -65, 19, -86, 50, -44, -97}}

o10 : List

i11 : time randomPoints(J, Strategy => GenericProjection)
-- used 0.364902 seconds

o11 = {{31, -15, 53, 41, 87, 9, -21, -43, -12, 85}}      
    \end{verbatim}
\end{quote}\vspace{-2em}
        }}
    \end{example}
To enable speed ups as indicated in the above discussion, the {\tt ProjectionAttempts} has been set to $0$ in the {\tt Default} strategy so that {\tt GenericProjection} is not opted for unless specified by user.
\begin{example}
        Finally, we allow our functions to extend our field.
        {{\small\color{blue}
        \begin{quote}
        \begin{verbatim}
i12 : time randomPoints(J, Strategy => LinearIntersection, ExtendField => true)
-- used 0.0612306 seconds

               5      4      3      2                                      5  
o12 = {{54, 14a  - 48a  - 56a  + 56a  - 13a + 48, -101, 56, -42, -27, - 26a  -
    -------------------------------------------------------------------------
       4      3      2
    43a  - 70a  - 96a  + 50a - 103, -29, 0, -37}}
        \end{verbatim}
    \end{quote}
        }}
    \end{example}

\begin{remark}[Finding a point]
	In the case of an absolutely irreducible hypersurface in $\bA^n_{\bF_q}$ (defined by $f$ say), there is significant discussion in the literature estimating lower bounds of number of rational points (see for instance, \cite{RationalKPointsSource,lang1954number,ghorpade2002number,cafure2006improved}) all of which point to the fact that there is ``good probability" of finding a rational point in this case when we intersect with a random line. Heuristically, we can make the following rough estimation.  We expect that each equation $f = \lambda$ for $\lambda \in \bF_q$ has approximately the same number of solutions.  Since each point on $\bF_q^n$ solves exactly one of these equations, we expect that $f = 0$ has approximately $q^{n-1}$ solutions, or in other words, our hypersurface has $q^{n-1}$-points.  
       Now, a random line $L$ has $q$ points. We want to find the probability that one of these points is  rational for $V(f)$.  We would expect that if these points are randomly distributed, then the probability that our line contains one of those points $1-(1-{1 \over q})^q$ which tends to $1-e^{-1} \approx 0.63$ for $q$ large.   Alternately, one can use the proof of \cite[Proposition 2.12]{bothmer2005quick} for a more precise statement.  For each point of $L$, we see that the probability that  the chosen point does not lie in the intersection, $L\cap V(f)$, is $1-\frac{1}{q}$. We then exhaust this search over all the points on $L$ to get the probability that there is indeed a successful intersection is $1-(1-\frac{1}{q})^{q}$. As $q$ gets larger, this value tends to $1-e^{-1}\approx 0.63$.

       Of course, there are certainly examples schemes over $\bF_q$ with no rational points at all, even for curves in $\bA^2$.
\end{remark}

\begin{remark}[Projecting to a hypersurface first]
    \label{rem.ProjectingToAHypersurfaceFirst}
    Suppose $X \subseteq \bA^n$ is an algebraic set.  In a number of existing algorithms, one first does a generic (or even not very generic) projection $h : \bA^n \to \bA^m$ and so that $h(X)$ is a hypersurface (at least set theoretically).  Then we find a point $x \in h(X)$ (say as above), and compute $h^{-1}(\{ x \})$ which is a linear space in $\bA^n$ that typically intersects $X$ in a rational point.   For example, this is done in {\tt randomKRationalPoint} in core Macaulay2.  Note that projecting to a hypersurface \emph{still is performing an intersection with a linear space}, but it just tries to choose the linear space more intelligently.

    However, while in a number of examples, doing this generic projection first can speed up computations (as opposed to simply intersecting with a linear space, since we have identified a linear space initially), there are numerous cases where computing this hypersurface $h(X)$ can be extremely slow.  This particularly appears in cases when one is computing successive minors to identify the locus where some variety is nonsingular.  Thus, by default, we disable the functionality to project to a hypersurface in the default strategy.

    On the other hand, instead of using a truly random linear space to intersect with, in the default strategy, we normally choose a linear space whose defining equations are particularly non-complicated (i.e., with many linear or binomial terms).  Such simple linear spaces are the ones considered in {\tt randomKRationalPoint} for instance.  In practice, this seems give approximately the same performance as projecting to a hypersurface, without the chance of the code getting hung up on the generic projection.
\end{remark}


\section{Two helper functions: {\tt dimViaBezout} and {\tt randomCoordinateChange}}{\label{helper}}

\subsection*{\tt dimViaBezout:} We thank the reviewer for pointing out that in most of the computations, computing the codimension of the given ideal, significantly slows down speed. Instead, we compute the dimension of $V(I)$ probabilistically using {\tt dimViaBezout}. This function takes as input an ideal $I$ in a polynomial ring over a field and intersects $V(I)$ with successively higher dimensional random linear spaces until there is an intersection.  For example, if $V(I)$ intersect a random line has a point, then we expect that $V(I)$ contains a hypersurface.  If there was no intersection, this function tries a $2$-dimensional linear space, and so on.  This speeds up many computations. The function also takes in optional inputs as described below:

\begin{itemize}
	\item {\tt DimensionIntersectionAttempts}: 
	This integer input denotes the number of linear spaces to try before moving to the next dimension.
	\item {\tt MinimumFieldSize}:
	If the ambient field is smaller than this integer value, it will automatically be replaced with an extension field. The user may set the {\tt MinimumFieldSize} to ensure that the field being worked over is big enough.  For instance, there are relatively few linear spaces over a field of characteristic $2$, and this can cause incorrect results to be provided. 
    \item {\tt Homogeneous}: 
        By default, the function uses one of two algorithms depending on whether the ideal is homogeneous.
        There are homogeneous examples where the non-Homogeneous method is faster.  Setting this to {\tt false} will force non-homogeneous computation even in homogeneous examples.  Setting it to true will force homogeneous computation, which may behave unexpectedly in non-homogeneous examples.
\end{itemize}

\begin{example} We illustrate the speed difference in this example.
	{\color{blue}{
			\begin{quote}
				\begin{verbatim}
				i3 : kk=ZZ/nextPrime 10^2;
				
				i4 : S=kk[y_0..y_14];
				
				i5 : I=minors(2,random(S^3,S^{5:-1}));
				
				o5 : Ideal of S
				
				i6 : elapsedTime dimViaBezout(I)
				-- 0.303629 seconds elapsed
				
				o6 = 7
				
				i7 : elapsedTime dim I
				-- 1.47916 seconds elapsed
				
				o7 = 7
				\end{verbatim}
			\end{quote}
	}}
\end{example}

As discussed earlier, one can always compute the codimension (say, $c$) of the input ideal using the output from the above function (or by existing command {\tt codim} beforehand), and provide it as an optional input {\tt Codimension => c} whenever calling {\tt randomPoints}. Alternatively, one can simply rely on any of the strategies for {\tt randomPoints} (see \Cref{strategydetails}) as they compute the dimension of $V(I)$ using {\tt dimViaBezout}, which significantly accelerates the computations.

\subsection*{\tt randomCoordinateChange:} 

This function takes a polynomial ring as an input and outputs the coordinate change map,
i.e. given a polynomial ring, this will produce a linear automorphism of the ring.  This function checks whether the map is an isomorphism by computing the Jacobian.

In some applications, a full change of coordinates is not desired, as it might cause code to run very slowly.  A binomial change of coordinates might be preferred, or we might only replace some monomials by other monomials.  
This is controlled with the following options.

\begin{itemize}
\item {\tt Replacement}: Setting {\tt Replacement => Full} will mean that coordinates are replaced by a general degree $1$ form.  If {\tt Replacement => Binomial}, the coordinates are only changed to binomials, which can be much faster for certain applications. If {\tt Homogeneous => false}, then there will be constant terms, and we view $mx + b$ as a binomial.
\item {\tt MaxCoordinatesToReplace}: The user can specify that only a specified number of coordinates should be non-monomial (assuming {\tt Homogeneous => true}).  
  This option is passed to {\tt randomCoordinateChange} by other functions that call it. 		
\item {\tt Homogeneous}:  Setting {\tt Homogeneous => false} will cause degree zero terms to be added to modified coordinates (including monomial coordinates).		
\end{itemize} 
%\par
%\vspace{-1em}    
\begin{example}  We demonstrate some of these options.
%
  {{\small\color{blue}
  \begin{quote}
\begin{verbatim}
	i3 : R = ZZ/11[x, y, z];
	
	i4 : randomCoordinateChange(R)
	
           ZZ
o4 = map(R,--[x, y, z],{4x + 5y - 5z, 3x - 4y - 3z, 4x})                     
           11
           
                    ZZ
o4 : RingMap R <--- --[x, y, z]
                    11
                    
	i5 : matrix randomCoordinateChange(R, MaxCoordinatesToReplace => 1)
	
	o5 = | x -x-4y-5z y |
	
	i6 : matrix randomCoordinateChange(R, MaxCoordinatesToReplace => 1, 
	Homogeneous => false)
	
	o6 = | x-3 z-5 -x+3y-4z+2 |
	
	i7 : matrix randomCoordinateChange(R, MaxCoordinatesToReplace => 1, 
	Replacement => Binomial)
	
	o7 = | y x+4z z |
\end{verbatim}
\end{quote}
    }}
  
\end{example}


\section{Other helper functions: {\tt genericProjection}, {\tt projectionToHypersurface}}\label{projectionfunctions}

We include two functions providing customizable projections. We describe them here. 


% \begin{enumerate}
\subsection{genericProjection} 
This function finds a random (somewhat, depending on options) generic projection of the ring or ideal.
The typical usages are as follows: 
\vspace{0.5em}

-- {\tt genericProjection(n, I)},

-- {\tt genericProjection(n, R)},


-- {\tt genericProjection(I)},  

-- {\tt genericProjection(R)} 

\vspace{0.5em}
\noindent where 
$I$ is an ideal 
in a polynomial ring, 
$R$ can denote a quotient of a polynomial ring and 
$n\in \mathbb{Z}$ is
an integer specifying how many dimensions to drop.  Note that this function makes no attempt to verify that the projection is actually generic with respect to the ideal.


This gives the projection map from $\mathbb{A}^N \mapsto\mathbb{A}^{N-n}$ and the defining ideal of the projection of $V(I)$. If no integer $n$ is provided then it acts as if $n = 1$. 


\begin{example}	
    We project a curve in $4$-space to one in $2$-space.
    ~~
  {{\small\color{blue}
  \begin{quote}
\begin{verbatim}
i1 : R = ZZ/5[x, y, z, w];

i2 : I = ideal(x, y^2, w^3 + x^2);

i3 : genericProjection(2, I)

            ZZ                                         2          2
o3 = (map(R,--[z, w],{- x - 2y - z, - y - 2z}), ideal(z  - z*w - w ))                   
            5
\end{verbatim}
\end{quote}
    }}
\end{example}\vspace{-1em}
Alternatively, instead of {$I$}, we may pass it a quotient ring.  It will then return the inclusion of the generic projection ring into the given ring, followed by the source of that inclusion.  It is essentially the same functionality as calling {\tt genericProjection(n, ideal R)} although the format of the output is slightly different. 

\begin{comment}~~
  
  {{\small\color{blue}
\begin{verbatim}
i1 : R = ZZ/5[x, y, z, w];

i2 : I = ideal(x, y^2, w^3 + x^2);

i3 : genericProjection(2, R/I)	

                            ZZ                          ZZ
                            --[z, w]                    --[z, w]                        
                R           5                           5
o3 = (map(----------------,--------,{2y - w, - y - z}), --------)                       
             2   3    2        4                            4
        (x, y , w  + x )      z                            z
        
o3 : Sequence
\end{verbatim}}
    }}
\end{comment}


This method works by calling {\tt randomCoordinateChange} (\Cref{helper}) before dropping variables.  It passes the options {\tt Replacement}, {\tt MaxCoordinatesToReplace}, {\tt Homogeneous} to that function.
%The user can also specify the same options as before to be passed on to the {\tt randomCoordinateChange} function.
\begin{comment}~~
  
  ~~
  
  {{\small\color{blue}
\begin{verbatim}
		
i2 : R = ZZ/5[x, y, z, w];

i3 : I = ideal(x, y^2, w^3 + x^2);

o3 : Ideal of R

i4 :genericProjection(2, I, Replacement => Binomial)

            ZZ                                   4
o4 = (map(R,--[z, w],{x + 2z, - 2y + w}), ideal w )                                     
            5
            
o4 : Sequence
\end{verbatim}}
    }}
\end{comment}

% \vspace{1em} 
\subsection{\tt projectionToHypersurface} This function creates a projection to a hypersurface. The typical usages are as follows: 

\vspace{0.5em}
-- {\tt projectionToHypersurface I},

-- {\tt projectionToHypersurface R} 
\vspace{0.5em}

\noindent where $I$ is an ideal in a polynomial ring, respectively, $R$ is a quotient of a polynomial ring. The output is
a list with two entries: the generic projection map and the ideal (respectively the ring).

It differs from {\tt genericProjection(codim I - 1, I)} as it only tries to find a hypersurface equation that vanishes along the projection, instead of finding one that vanishes exactly at the projection.  This can be faster and can be useful for finding points.   If we already know the codimension is {\tt c}, we can set {\tt Codimension=>c} so the function does not compute it. 

%\vspace{1em}
\begin{comment}~~

  ~~
  
  {{\small\color{blue}
\begin{verbatim}
	i2 : R = ZZ/5[x, y, z];
	
	i3 : I = ideal(random(3, R) - 2, random(2, R));
	
	o3 : Ideal of R
	
	i4 : projectionToHypersurface(I)
	
	            ZZ                             4    3      2 2     4
	o4 = (map(R,--[y, z],{x + 2y, 2y}), ideal(y  + y z - 2y z  - 2z  - y - z))
	            5
	
    o4 : Sequence
	i5 : projectionToHypersurface(R/I)
	
	
	
                           	                    R                    
	o5 = (map(--------------------------------------------------------------
	    3       2     3            2        2       2       2     
	(- x  - 2x*y  - 2y  - x*y*z + y z - 2x*z  + 2y*z  - 2, x  - 2x
	-------------------------------------------------------------------
                                    ZZ                       
	                                    --[y, z]                 
	                                    5                       
	--------------------,----------------------------------------------
	      2               4 2    3 3       5     6     3       2     3 
	*y - y  + x*z - y*z) y z  + y z  - 2y*z  + 2z  + 2y  + 2y*z  + 2z  
	-------------------------------------------------------------------
                    	       ZZ
                	           --[y, z]
	                            5
	---,{x - z, y}), -------------------------------------------------)
	                  4 2    3 3       5     6     3       2     3
	- 1              y z  + y z  - 2y*z  + 2z  + 2y  + 2y*z  + 2z  - 1
	
	o5 : Sequence
	\end{verbatim}
    }}
\end{comment}


% \end{enumerate}




\section{An application:  {\tt findANonZeroMinor}, {\tt extendIdealByNonZeroMinor}}

As mentioned in the introduction, the two functions in this section will provide further tools for computing singular locus, in addition to those available in the package {\tt FastLinAlg}. 

\subsection{\tt findANonZeroMinor:}\label{findANonZeroMinor} The typical usage of this function is as follows: 

\vspace{0.5em}
-- {\tt findANonZeroMinor(n, M, I)} 
\vspace{0.5em}

\noindent where $I$ is an ideal
in a polynomial ring over {\tt QQ} or {\tt ZZ/p} for $p$ prime, $M$ is a matrix
over the polynomial ring and $n\in \mathbb{Z}$ denotes the size of the minors of interest.


The function outputs the following:

-- randomly chosen point $P$ in $V(I)$ which it finds using {\tt randomPoints}.

-- the indexes of the columns of $M$ that stay linearly independent upon plugging $P$ into $M$, 

-- the indices of the linearly independent rows of the matrix extracted from $M$ in the above step, 

-- a random $n\times n$ sub-matrix of $M$ that has full rank at $P$.

The user may also provide the following additional information: 

\begin{description}
\item[\tt Strategy => Symbol] To specify which strategy to use when calling {\tt randomPoints} (see \Cref{strategydetails}).
  
\item[\tt Verbose => Boolean]
  Set the option {\tt Verbose => true} to turn on verbose output.  This may be useful in debugging or in determining why a computation is running slowly.
  
\item[\tt Homogeneous => Boolean] (see \Cref{helper})
  
\item[\tt MinorPointAttempts => ZZ] 
  This controls how many points at which to check the rank of the matrix.
\end{description}
%\vspace{0.25em}
\begin{example}
  We demonstrate this function.
  {{\small\color{blue}
  \begin{quote}
\begin{verbatim}
		i3 : R = ZZ/5[x, y, z];
		
		i4 : I = ideal(random(3, R) - 2, random(2, R))
		
		              3     2        2    3    2              2        2       2
		o4 = ideal (2x  - 2x y + 2x*y  + y  + x z - 2x*y*z + y z - 2x*z  + 2y*z 
		-------------------------------------------------------------------
		   3                      2
		- z  - 2, - 2x*y - x*z - z )
		
		o4 : Ideal of R
		
		i5 :  M = jacobian(I)
		
		o5 = {1} | x2+xy+2y2+2xz-2yz-2z2   -2y-z |
		     {1} | -2x2-xy-2y2-2xz+2yz+2z2 -2x   |
		     {1} | x2-2xy+y2+xz-yz+2z2     -x-2z |
		
		             3       2
		o5 : Matrix R  <--- R
		
		i6 : findANonZeroMinor(2, M, I, Strategy => GenericProjection)
		
		o6 = ({-2, 1, 1}, {0, 1}, {0, 1}, {1} | x2+xy+2y2+2xz-2yz-2z2   -2y-z |)
		{1} | -2x2-xy-2y2-2xz+2yz+2z2 -2x   |
\end{verbatim}
\end{quote}  
    }}
\end{example}

\subsection{\tt extendIdealByNonZeroMinor:}\label{extendIdealByNonZeroMinor} The typical usage is 

\vspace{0.5em}
-- {\tt extendIdealByNonZeroMinor(n, M, I)} 

\vspace{0.5em}
\noindent where $n,M,I$ are same as before. This function finds a submatrix of size $n\times n$ using \\{\tt findANonZeroMinor};  
it extracts the last entry of the output, finds its determinant and
adds it to the ideal $I$, thus extending $I$.   It has the same options as {\tt findANonZeroMinor}.


% \vspace{1em}


%\vspace{1em}
\begin{comment}~~
  
  ~~
  
  {{\small\color{blue}
\begin{verbatim}
i2 : R = ZZ/5[x, y, z];
i3 : I = ideal(random(3, R) - 2, random(2, R))
                3    2       2     3     2              2       2     3
o3 = ideal (- 2x  + x y - x*y  - 2y  - 2x z + 2x*y*z + y z + y*z  - 2z 
-------------------------------------------------------------------
         2            2                 2
- 2, - 2x  - 2x*y + 2y  + x*z - 2y*z - z )
o3 : Ideal of R
i4 : M = jacobian(I)
o4 = {1} | -x2+2xy-y2+xz+2yz    x-2y+z   |
     {1} | x2-2xy-y2+2xz+2yz+z2 -2x-y-2z |
     {1} | -2x2+2xy+y2+2yz-z2   x-2y-2z  |
             3       2
o4 : Matrix R  <--- R
i5 : extendIdealByNonZeroMinor(2, M, I, Strategy => LinearIntersection)
                3    2       2     3     2              2       2     3
o5 = ideal (- 2x  + x y - x*y  - 2y  - 2x z + 2x*y*z + y z + y*z  - 2z 
-------------------------------------------------------------------
         2            2                 2   3    2        2    3  
- 2, - 2x  - 2x*y + 2y  + x*z - 2y*z - z , x  + x y + 2x*y  - y  +
-------------------------------------------------------------------
  2       2    3
2x z + y*z  - z )
o5 : Ideal of R
\end{verbatim}}	
    }}	
\end{comment}

One can use this function to show that rings are $(R_1)$, that is, regular in codimension $1$.

\begin{example}
    Consider the following example (which is $(R_1)$) where computing the dimension of the singular locus takes around $30$ seconds as there are $15500$ minors of size $4 \times 4$ in the associated $7 \times 12$ Jacobian matrix.  However, we can use this function to quickly find interesting minors.
  {{\small\color{blue}
  \begin{quote}
\begin{verbatim}
		i2 : T = ZZ/101[x1, x2, x3, x4, x5, x6, x7];
		
		i3 : I =  ideal(x5*x6-x4*x7,x1*x6-x2*x7,x5^2-x1*x7,x4*x5-x2*x7,x4^2-x2*x6,x1*x4-x2*x5,
		x2*x3^3*x5+3*x2*x3^2*x7+8*x2^2*x5+3*x3*x4*x7-8*x4*x7+x6*x7,
		x1*x3^3*x5+3*x1*x3^2*x7+8*x1*x2*x5+3*x3*x5*x7-8*x5*x7+x7^2,
		x2*x3^3*x4+3*x2*x3^2*x6+8*x2^2*x4+3*x3*x4*x6-8*x4*x6+x6^2,
		x2^2*x3^3+3*x2*x3^2*x4+8*x2^3+3*x2*x3*x6-8*x2*x6+x4*x6,
		x1*x2*x3^3+3*x2*x3^2*x5+8*x1*x2^2+3*x2*x3*x7-8*x2*x7+x4*x7,
		x1^2*x3^3+3*x1*x3^2*x5+8*x1^2*x2+3*x1*x3*x7-8*x1*x7+x5*x7);
		
		o3 : Ideal of T
		
		i4 : M = jacobian I;
		
		             7       12
		o4 : Matrix T  <--- T
		
		i5 : i = 0;
		
		i6 : J = I;
		
		o6 : Ideal of T
		
		i7 : elapsedTime(while (i < 10) and dim J > 1 do (
                                i = i + 1; 
                                J = extendIdealByNonZeroMinor(4, M, J)));
		-- 0.903328 seconds elapsed
		
		i8 : dim J
		
		o8 = 1
		
		i9 : i
		
		o9 = 5
\end{verbatim}
\end{quote}
    }}    
    \noindent
    In this particular example, there tend to be about $5$ associated primes when adding the first minor to $J$, and so one expects about $5$ steps as each minor will typically eliminate one of those primes.
\end{example} 

There is some similar functionality computing partial Jacobian ideals obtained via heuristics (as opposed to actually finding rational points) in the package {\tt FastLinAlg}, see \cite{FastLinAlgSource}.  That package now uses the functionality contained here in {\tt RandomRationalPoints} in some of its functions.

% \end{enumerate}

\bibliographystyle{alpha}
\bibliography{MainBib}

\end{document}
